\chapter{Conclusion}\label{andfinally}
The last chapter of a well-structured report reviews what's been done and
mentions the main open questions.  It's the part a typical reader looks
at second, after the Introduction, and it should be just as appealing.
\par
This template is provided only to help you start with \LaTeX\ and to
exemplify good structure. When you've got some practice, you'll want to
alter it to suit your topic and your taste --- and you are free to do
that.
\begin{itemize}
    \item\verb+\documentclass[a4paper,twocolumn,10pt]{article}+
    gives an attractive alternative framework, for example.\par Or you
may prefer \verb+\documentclass{memoir}+ which has an excellent
manual \cite{MEM} that includes a general introduction to typography.
    \item Try to resist the temptation to make margins narrower
    and lines longer --- it may save a sheet or two of paper but it
    can make the text hard to read. Indeed, typesetting professionals 
    quote a maximum of 66 characters per line as an ideal for easy reading
          \cite[Sec.~5.2.2]{NSS}.
    \item proof-read carefully --- not only to get the best from \LaTeX\
but also to polish grammar, spelling and punctuation \cite{ESL}. See
App.~\ref{app:typo} for hints on how to avoid common pitfalls.
\end{itemize}
Good layout and use of \LaTeX, plus fluent and grammatical writing, will
together give a very good impression.
\par
Top tip --- write what you think is the final version, then put it
away in a drawer and come back to it a week or so later. With fresh
eyes you'll see many potential improvements. This works for anything you
prepare, not just a project report. It needs forward planning of
course!
\section*{Acknowledgements}
\addcontentsline{toc}{chapter}{\numberline{}Acknowledgements}
Here's the place for thanks to anyone who particularly helped you.
Don't go too far OTT --- try to keep some dignity. This isn't the 
Oscars.

