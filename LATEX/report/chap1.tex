\chapter{Introduction}
The first chapter of a well-structured report is always an
introduction, setting the scene with motivation and context (as in
Sec.~\ref{intro}) and then looking ahead to summarise what's in the
rest of the report (as in Sec.~\ref{intro:contents}). It's the
bit that readers look at first --- {\em so make sure it hooks them!}
%
\section{Context and motivation}\label{intro}
This is a template for \LaTeX\ project reports in the Department of
Mathematical Sciences. It shows a good overall structure for the
printed document, and shows how to construct it with a master file
(\texttt{report.tex}) plus subsidiary files (\texttt{chap1.tex},
\dots, \texttt{app1.tex}, \dots, \texttt{biblio.tex}).
\par
At the same time, features of the current version of \LaTeX\ (\LaTeXe)
are illustrated --- such as mathematical expressions, numbering and
cross-referencing, bibliography and citations, graphics and tables.
Comparison of the source files with the printer-ready document will
answer a few FAQs: \Quote{How can I do \dots\ in \LaTeX?}.
\par
However, this is {\em not} a textbook on \LaTeX\ --- for that, use the
\lq\nss\rq\ notes by Oetiker \etal\ \cite{NSS}. They are written for
novices, and are a pleasure to read. They are available free on-line,
and are kept up-to-date. The \LaTeX\ book at \textsl{Wikipedia}
\cite{WL} includes the \nss\ material and is good for reference too.
Access these via the \LaTeX\ resources page \cite{LAT}.
\par
For more advanced features see \eg\lq\comp\rq\ \cite{MG}.
\par
Well-meant advice on \LaTeX\ for report-writing and poster-making is
available\footnote{From  \texttt{bob.johnson@dur.ac.uk}} in room CM315,
where there are reference copies of both \comp\ \cite{MG} and
\textsl{The Graphics Companion} \cite{GRM}.
\par
Even if you are misguided enough \cite{AC} to prepare your report in
\textsl{Word}, this template at least exemplifies a good structure ---
and gives advice about references and help with typography.
%
\section{Contents}\label{intro:contents}
The main body of this report is divided as follows.
\par
Chap.~\ref{sec:formulas} has some examples of mathematics, then
Chap.~\ref{sec:graphics} deals with graphics and includes
Sec.~\ref{sec:tables} about tables. The Conclusion, in
Chap.~\ref{andfinally}, summarises what's been achieved, the open
questions and what could be done next.
\par
Then comes the Bibliography, listing all sources of material, data and
computer programs used, \etc. Its construction is explained in
\cite[Sec.~4.2]{NSS} and there's more about it in App.~\ref{app:refs}.
\par
Otherwise appendices typically hold basic background theory, or
additional or similar examples, or longer proofs (App.~\ref{app:proofs})
 --- anything you need but which would hold up the main flow of the
story. You could also use an appendix for listings of any computer
programs that you've written (App.~\ref{app:programs}).
\par
Here there's also brief advice on grammar and typography
(App.~\ref{app:typo}).
%
\section{Using a PC}\label{sec:intro:parttwo}
You may want \LaTeX\ on your own computer.
\par
A popular version of \TeX/\LaTeX\ for a Windows PC is \Quote{MikTex}
\cite{MKT}. \miktex\ is free, and is available online for 
download \cite{LAT} or locally on a DVD from
\texttt{bob.johnson@dur.ac.uk}, room CM315. It's also installed on the
ITS Networked PC Service under \texttt{Programs | Miscellaneous}.
\par
To use \miktex\ easily you need a dedicated
editor or IDE\footnote{Computer scientists say \lq Integrated 
Development Environment'.}. A good one is \Quote{WinEdt} \cite{WDT}, 
which runs under all recent versions of Microsoft Windows and integrates 
well with \miktex. \textsl{WinEdt} is free for 31 days; to use
it thereafter costs about \$40.
\par 
Completely free rivals include \Quote{TeXnicCenter} \cite{TXC}, which is
provided on the \miktex\ DVD and also set up on the ITS Networked PC
Service under \texttt{Programs | Miscellaneous | MikTeX}.
\par
Others include \eg \Quote{WinShell} \cite{WSH}, but are untried.
\par
All these editors/IDEs have a familiar style of graphical user-interface
with a toolbar and pull-down menus for all the common tasks involved in
editing source files, running \LaTeX\ and viewing the results.
